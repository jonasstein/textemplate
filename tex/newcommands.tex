% [\newcommand*] defines a [command], that is not, in TeX terms, long.


\newcommand{\comment}[1]{} % comment out for debugging with \comment{}

%\newcommand*{\person}[1]{\hbox{#1}\sindex[person]{#1}}
\newcommand*{\person}[1]{\textsc{{#1}}} % perhaps make some hacks with that later..
\newcommand*{\device}[1]{\textsc{{#1}}}


\newcommand*{\Cgeo}{\mathrm{C_{geo}}} % Geometric Capacity

%\newcommand*{\Cref}{C_{\textrm{ref}}} % in use by cleverref!!!
\newcommand*{\Qref}{Q_{\text{ref}}}

\providecommand\t{}

% H. Voß Mathematiksatz 2. Aufl. S. 76
\newcommand*\diff{\mathop{}\!\mathrm{d}}

%\usepackage{paralist, dcolumn, ragged2e}
% H. Voß Tabellen S. 205

%\newcolumntype{v}[1]{>{\RaggedRight\hspace{0pt}}p{#1}}

%http://tex.stackexchange.com/questions/64631/perp-and-parallel-with-equal-height
\newcommand{\myparallel}{{\mkern3mu\vphantom{\perp}\vrule depth 0pt\mkern2mu\vrule depth 0pt\mkern3mu}}

% \PreviewMacro[{{}}]{\ce} %preview, if mhchem does not work

